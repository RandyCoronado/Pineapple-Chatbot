\documentclass[11pt]{article}

\usepackage[margin=1in]{geometry}
\usepackage{hyperref}
\usepackage{longtable}
\usepackage{array}
\usepackage{enumitem}
\usepackage{setspace}
\usepackage{graphicx}

\setlength{\parskip}{0.6em}
\setlength{\parindent}{0pt}
\setcounter{secnumdepth}{3}

\begin{document}

%----------------- TITLE PAGE -----------------%
\begin{titlepage}
    \centering
    \vspace*{3cm}

    {\Large \textbf{Software Design Document}\par}
    \vspace{0.5cm}
    {\LARGE \textbf{Pineapple Chatbot}\par}

    \vspace{1.5cm}
    {\large Prepared by:\par}
    {\large Kuong Thong, Enrique Castillo, Julio Aguilar, Jose Lopez, Yilin Ruan\par}

    \vfill
    {\large CSULA ITS Department\par}
    \vspace*{1cm}
\end{titlepage}

%----------------- VERSION HISTORY -----------------%
\section*{Version History}

\begin{longtable}{>{\raggedright\arraybackslash}p{3cm}%
                  >{\raggedright\arraybackslash}p{3cm}%
                  >{\raggedright\arraybackslash}p{8cm}}
\textbf{Version} & \textbf{Date} & \textbf{Summary of Changes} \\
\hline
apple       & 12/4/18  & Added Introduction sections (1.1--1.2). \\
banana      & 12/5/18  & Added early DFD diagrams and transferred initial design content. \\
cantaloupe  & 12/6/18  & Added database design and sections 8--9.1. \\
dates       & 12/7/18  & Completed document draft and added validation sections. \\
elderberry  & 1/3/19   & Updated DFDs to match current model. \\
farkleberry & 3/13/19  & Updated feature descriptions and removed outdated content. \\
grapefruit  & 3/22/19  & Cleaned document structure and updated sections 5--7. \\
huckleberry & 9/25/19  & Updated crawling approach. \\
jackfruit   & 11/15/19 & Removed Amazon Lex dependency. \\
kiwi        & 12/2/19  & Updated models to reflect spaCy usage. \\
\end{longtable}

\newpage
\tableofcontents
\newpage


%----------------- 1. INTRODUCTION -----------------%
\section{Introduction}

\subsection{Purpose}
Pineapple Chatbot is an informational chatbot for a standalone web application that helps users
find information related to the CSULA website. The chatbot processes user queries, responds
with useful answers, and can receive user feedback.

\subsection{Document Conventions}
This document follows a structured SDD format with enumerated sections, module definitions,
and design descriptions.

\subsection{Intended Audience}
This document is intended for developers, testers, designers, project managers, and documentation staff.

\subsection{System Overview}
The chatbot crawls university webpages, stores extracted information in a database, and uses
natural-language processing to interpret user queries. It returns answers as text, links, or
summaries.



%----------------- 2. DESIGN CONSIDERATIONS -----------------%
\section{Design Considerations}

\subsection{Assumptions \& Dependencies}
\begin{itemize}[nosep]
    \item Backend built using Python.
    \item Libraries: MySQL Connector, BeautifulSoup, nltk, ChatterBot, spaCy.
    \item Additional dependencies: NodeJS, React (front-end).
\end{itemize}

\subsection{Constraints}
\begin{itemize}[nosep]
    \item Early versions lack spell-check, web abstraction, and high-accuracy NLP.
    \item System quality depends on crawled data; vague queries may yield incorrect results.
\end{itemize}

\subsection{Goals}
\begin{itemize}[nosep]
    \item Provide helpful, conversational responses.
    \item Keep design simple (KISS principle).
    \item Provide detail without overwhelming users.
    \item Maintain high user satisfaction.
\end{itemize}

\subsection{Development Methods}
Pair programming and iterative testing were used to refine modules and improve accuracy.



%----------------- 3. ARCHITECTURAL STRATEGY -----------------%
\section{Architectural Strategy}

Python is used for crawling, NLP, and chatbot logic due to its open-source ecosystem.  
MySQL stores processed data because it integrates well with ChatterBot's SQL backends.  
The chatbot runs inside a web application built with React, which manages user input and output.



%----------------- 4. SYSTEM ARCHITECTURE -----------------%
\section{System Architecture}

\subsection{Level 0 DFD}
\begin{figure}[h!]
    \centering
    \includegraphics[width=0.8\textwidth]{DFD_Level0_PineappleChatbot}
    \caption{DFD Level 0 for Pineapple Chatbot}
\end{figure}

\subsection{Level 1 DFD}
\begin{figure}[h!]
    \centering
    \includegraphics[width=\textwidth]{DFD_Level1_PineappleChatbot}
    \caption{DFD Level 1 for Pineapple Chatbot}
\end{figure}

\subsection*{Modules}
\begin{enumerate}
    \item Input Module – Collects user inputs.
    \item Preprocessor Module – Cleans and normalizes text.
    \item Logic Module – Identifies user intent and selects a response.
    \item Storage Module – Stores and retrieves crawled data.
    \item Output Module – Formats and returns the answer to the user.
\end{enumerate}

\subsection*{Data Flow Summary}
User query → Preprocessing → Intent recognition → Keyword lookup → Data retrieval → Response → User.



%----------------- 5. POLICIES AND TACTICS -----------------%
\section{Policies and Tactics}

\subsection{Tools \& Technologies}
ChatterBot, React, Python, MySQL.

\subsection{Requirements Traceability}
Crawled data is backed up regularly.

\subsection{Testing Approach}
User testing and QA validate accuracy and relevance of chatbot responses.

\subsection{Engineering Trade-offs}
Campus websites store large amounts of information; storing more data increases coverage but
slows query speed.

\subsection{Coding Standards}
Standard Python and JavaScript conventions are followed; version control is used throughout.

\subsection{Source Code Organization}
\begin{itemize}[nosep]
    \item \textbf{React App}: User interface
    \item \textbf{Python Backend}: Chatbot logic and server
    \item \textbf{Database}: MySQL storage
\end{itemize}

\subsection{System Build Instructions}
\begin{enumerate}
    \item Start VM
    \item Run React frontend (npm start)
    \item Run Python backend (python manage.py runserver)
    \item Access chatbot via CSULA server interface
\end{enumerate}



%----------------- 6. DETAILED SYSTEM DESIGN -----------------%
\section{Detailed System Design}

\subsection{Input Module}
Responsibilities: Convert user input into Statement objects.  
Constraints: Must handle text and minimal structured formats.  
Interactions: Passes processed input to Logic Module.

\subsection{Preprocessor Module}
Responsibilities: Clean whitespace, remove HTML artifacts, convert Unicode to ASCII.  
Limitations: Must preserve important keywords.

\subsection{Storage Module}
Responsibilities: Store responses and crawled data.  
Technologies: MySQL; SQLStorageAdapter recommended.  
Limitations: Requires credentials; large datasets may slow queries.

\subsection{Logic Module}
Responsibilities include:
\begin{itemize}[nosep]
    \item Recognize intent
    \item Identify keywords
    \item Select best-matched response
    \item Optionally evaluate time-based or mathematical queries
\end{itemize}

\subsection{Output Module}
Responsibilities: Format and return final response to user through React UI.



%----------------- 7. LOWER-LEVEL COMPONENT DESIGN -----------------%
\section{Lower-Level Component Design}

\subsection{Input Module Components}
\begin{itemize}[nosep]
    \item Creates Statement objects from input
    \item Accepts text, dictionary, or Statement formats
\end{itemize}

\subsection{Output Module Components}
\begin{itemize}[nosep]
    \item Returns Statement objects as responses
    \item Reads from Storage Module
\end{itemize}

\subsection{Storage Module Components}
\begin{itemize}[nosep]
    \item CRUD operations for stored responses
    \item Requires DB credentials
    \item SQL module is primary implementation
\end{itemize}

\subsection{Logic Module Components}
\begin{itemize}[nosep]
    \item Compares user statements with known statements
    \item Uses similarity thresholds and exclusion rules
    \item Selects closest-matching response
\end{itemize}



%----------------- 8. DATABASE DESIGN -----------------%
\section{Database Design}

A MySQL database stores the crawled website data.

\subsection*{Key Table: \texttt{termUrlKeywords}}
Approximately 300,000 records.
\begin{itemize}[nosep]
    \item Columns: \texttt{term}, \texttt{URL}, \texttt{keywords}
\end{itemize}

Additional tables contain faculty information and classifier labels.

Database enables fast keyword matching, URL lookup, and long-term storage.



%----------------- 9. USER INTERFACE -----------------%
\section{User Interface}

\subsection{Overview}
\begin{itemize}[nosep]
    \item Simple chat-style interface built with React
    \item Input box for queries
    \item Response window for answers and links
\end{itemize}

\subsection{Future Enhancements}
\begin{itemize}[nosep]
    \item Richer UI components
    \item Improved accessibility
\end{itemize}



%----------------- 10. VALIDATION & VERIFICATION -----------------%
\section{Requirements Validation \& Verification}

\begin{itemize}[nosep]
    \item Unit tests for crawler, NLP, and storage
    \item Integration tests for full query flow
    \item User testing for clarity and accuracy
\end{itemize}



%----------------- 11. GLOSSARY -----------------%
\section{Glossary}

\begin{itemize}[nosep]
    \item \textbf{Statement}: Processed user input
    \item \textbf{Intent}: Interpreted purpose of a query
    \item \textbf{Similarity Score}: Metric to select the closest matching response
\end{itemize}



%----------------- 12. REFERENCES -----------------%
\section{References}

Original Pineapple Chatbot documents, tutorials for Python NLP libraries, and MySQL documentation.

\end{document}
