\documentclass[12pt]{article}
\usepackage{geometry}
\geometry{margin=1in}
\usepackage{setspace}
\usepackage{titlesec}
\usepackage{hyperref}
\usepackage{enumitem}

\title{Software Requirements Specification \\ \large{Pineapple Chatbot} \\ Version 2.0}
\author{Kuong Thong, Enrique Castillo, Julio Aguilar, Jose Lopez, Yilin Ruan \\ CSULA ITS Department}
\date{March 22, 2020}

\begin{document}

\maketitle
\newpage

\tableofcontents
\newpage

\section*{Revision History}
\begin{center}
\begin{tabular}{|c|c|p{7cm}|c|}
\hline
\textbf{Name} & \textbf{Date} & \textbf{Description} & \textbf{Version} \\
\hline
Initial Draft & TBD & Initial skeleton draft created for Snapshot 1 & 1.0 \\
\hline
Expanded Draft & TBD & Added detailed requirements, Bayesian model, database structure, UI, updated constraints & 2.0 \\
\hline
\end{tabular}
\end{center}
\newpage

\section{Introduction}

\subsection{Purpose}
The Pineapple Chatbot provides general information extracted from pages across the 
\texttt{calstatela.edu} domain. Version 2.0 expands the chatbot’s functionality by adding Bayesian 
classification, improved NLP accuracy, early UI elements, and a transition from static links to a 
structured database.

\subsection{Intended Audience and Reading Suggestions}
This document is intended for developers, testers, marketing staff, project managers, and documentation authors.  
Readers seeking to understand how the chatbot works should read the entire document from top to bottom.

Testers should review the requirement list and data flow diagrams.  
Customers may review the capabilities of the Pineapple Chatbot.  
Investors may examine the product description and future goals.

\subsection{Product Scope}
The Pineapple Chatbot uses NLP techniques (primarily spaCy) and simple statistical models to 
interpret user intent and provide the most relevant webpage or text-based response.

Version 2.0 includes:
\begin{itemize}
\item Entity extraction using spaCy
\item Intent identification
\item Expanded link scoring model (Bayesian + keyword scoring)
\item Early-stage user interface
\item Integration with a preliminary URL database
\item Improved handling of unclear queries
\item Queueing of user inputs
\end{itemize}

\subsection{Definitions, Acronyms, and Abbreviations}
\begin{itemize}
\item \textbf{Bayesian Model} – Statistical classifier comparing keywords with category distributions.
\item \textbf{Keyword Reduction} – Removing stop-words to isolate meaningful terms.
\item \textbf{Category Boosting} – Increasing link scoring weight based on category match.
\item \textbf{NER} – Named Entity Recognition.
\end{itemize}

\subsection{References}
\begin{itemize}
\item spaCy Documentation
\item Python Standard Library
\item Naive Bayes text classification examples
\item CSULA website structure and categories
\end{itemize}

\newpage
\section{Overall Description}

\subsection{Product Perspective}
Pineapple Chatbot is a standalone web-based chatbot connected to backend NLP models.  
It currently runs locally with minimal UI; future versions will use a hosted deployment.

\subsection{Product Functions}

\subsubsection*{User Functions}
\begin{itemize}
\item Submit questions through the web interface
\item Receive chatbot replies through chat bubbles
\item Receive links or direct answers for professor-related queries
\end{itemize}

\subsubsection*{System Functions (New)}
\begin{itemize}
\item Bayesian classification of queries
\item Keyword extraction and stop-word filtering
\item Category matching
\item URL ranking using hybrid scoring
\item URL storage and retrieval from SQL database
\end{itemize}

\subsection{User Classes and Characteristics}
\textbf{General Users:} Students using the chatbot for quick campus resource lookup.  
\textbf{Developers:} Maintain NLP code, database structure, and UI logic.  
\textbf{Testers:} Validate classification accuracy and link relevance; create TestRail cases.

\subsection{Operating Environment}
\begin{itemize}
\item Python 3.x
\item spaCy NLP models
\item Web server (Node, Flask)
\item MySQL or SQLite database
\item Chrome, Firefox, Edge
\end{itemize}

\subsection{Design and Implementation Constraints}
\begin{itemize}
\item English-only supported
\item Entity accuracy limited by spaCy model
\item Website not mobile-optimized
\item VPN may be required for hosting
\item Link dataset must be updated periodically
\item Response time depends on DB + model ranking
\end{itemize}

\subsection{User Documentation}
No user manuals yet. Chatbot will be published on the CSULA website.

\subsection{Assumptions and Dependencies}
\begin{itemize}
\item CSULA website structure remains consistent
\item spaCy NER model can detect PERSON, ORG, etc.
\item Database is reachable
\item User questions use general English grammar
\end{itemize}

\newpage
\section{External Interface Requirements}

\subsection{User Interfaces}
Version 2.0 includes a basic web UI:

\textbf{Components:}
\begin{itemize}
\item Text input field
\item Send button
\item Scrollable chat window
\item Chat bubbles
\end{itemize}

\textbf{Behavior:}
\begin{itemize}
\item Auto-scroll
\item Loading indicator
\item Responses may include text + clickable links
\end{itemize}

\subsection{Hardware Interfaces}
\begin{itemize}
\item Standard laptop/desktop
\item Internet access
\end{itemize}

\subsection{Software Interfaces}
\begin{itemize}
\item spaCy NLP library
\item MySQL database
\item REST API returning JSON
\item Python NLP backend
\end{itemize}

\subsection{Communications Interfaces}
Internet access required. VPN may be required.  
Internet Explorer is not supported (Chrome, Firefox, Edge recommended).

\newpage
\section{Requirements Specification}

\subsection{Functional Requirements}

\subsubsection*{1. Input Module}
\begin{enumerate}[label=1.\arabic*]
\item System shall accept user text input.
\item Module should allow user feedback on answers.
\item System may allow multi-sentence input.
\end{enumerate}

\subsubsection*{2. Output Module}
\begin{enumerate}[label=2.\arabic*]
\item Display user’s question back to them.
\item Include a clickable link where available.
\item Display error message for unknown queries.
\end{enumerate}

\subsubsection*{3. Logic Module}
\begin{enumerate}[label=3.\arabic*]
\item Parse user input for keywords.
\item Recognize intent using spaCy.
\item Respond appropriately even to unclear input.
\item Send keywords and intent to storage module.
\item Receive list of URLs from storage module.
\item Determine the most relevant URL.
\item Send entity, intent, keywords to data extraction module.
\end{enumerate}

\subsubsection*{4. Data Extraction Module}
\begin{enumerate}[label=4.\arabic*]
\item Extract correct information from URLs.
\item Send extracted information to output module.
\item Parse and organize text; use spaCy for labeling.
\end{enumerate}

\subsubsection*{5. Storage Module}
\begin{enumerate}[label=5.\arabic*]
\item Send database queries.
\item Receive responses.
\item Return data to logic module.
\end{enumerate}

\subsection{External Interface Requirements}
\begin{itemize}
\item Web input must support English text.
\item URL responses must be HTML-safe.
\item Chat must support sequential message queues.
\end{itemize}

\subsection{Logical Database Requirements}

\textbf{Database Tables:}

\begin{itemize}
\item \textbf{urls}
\begin{itemize}
\item id
\item url
\item keywords
\item category
\end{itemize}

\item \textbf{categories}
\begin{itemize}
\item category\_id
\item name
\item keyword\_distribution
\end{itemize}
\end{itemize}

\textbf{Requirements:}
\begin{itemize}
\item Full-text search support
\item Batch inserts allowed
\item At least 500+ URL entries
\end{itemize}

\subsection{Design Constraints}
\begin{itemize}
\item spaCy model loads at startup
\item Queries must finish under 3 seconds
\item Category scoring accuracy 80–90\%
\end{itemize}

\newpage
\section{Other Nonfunctional Requirements}

\subsection{Performance Requirements}
\begin{itemize}
\item System response time: 1–3 seconds
\item DB query time: under 200ms
\item UI remains responsive during backend calls
\end{itemize}

\subsection{Safety Requirements}
Feedback system must protect user identity.  
Chatbot must not expose sensitive information or unsafe guidance.

\subsection{Security Requirements}
\begin{itemize}
\item Sanitize all input
\item DB requires authentication
\item HTTPS required once deployed
\item Admin interface requires login
\end{itemize}

\subsection{Software Quality Attributes}
\begin{itemize}
\item Usability: Simple chat interface
\item Reliability: Consistent NLP responses
\item Maintainability: Modular architecture
\item Scalability: Database expandable
\end{itemize}

\subsection{Business Rules}
\begin{itemize}
\item Bot must prioritize official CSULA info
\item All answers must come from \texttt{calstatela.edu}
\item Professor data must remain accurate
\end{itemize}

\section{Other Requirements}
Not applicable.

\begin{center}
\textbf{END OF VERSION 2.0}
\end{center}

\end{document}
