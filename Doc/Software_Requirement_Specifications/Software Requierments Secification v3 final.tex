\documentclass[12pt]{article}
\usepackage{geometry}
\geometry{margin=1in}
\usepackage{setspace}
\usepackage{titlesec}
\usepackage{hyperref}
\usepackage{enumitem}

\title{Software Requirements Specification \\ \large{Pineapple Chatbot} \\ Version 3.0}
\author{Kuong Thong, Enrique Castillo, Julio Aguilar, Jose Lopez, Yilin Ruan \\ CSULA ITS Department}
\date{December 10, 2025}

\begin{document}

\maketitle
\newpage

\tableofcontents
\newpage

\section*{Revision History}
\begin{center}
\begin{tabular}{|c|c|p{7cm}|c|}
\hline
\textbf{Name} & \textbf{Date} & \textbf{Description} & \textbf{Version} \\
\hline
Initial Draft & 10/15/25 & Initial skeleton draft created for Snapshot 1 & 1.0 \\
\hline
Expanded Draft & 11/5/25 & Added detailed requirements, Bayesian model, database structure, UI details & 2.0 \\
\hline
Final Release & 12/10/25 & Completed all sections, refined requirements, added glossary and references & 3.0 \\
\hline
\end{tabular}
\end{center}

\newpage
\section{Introduction}

\subsection{Purpose}
The Pineapple Chatbot provides general information extracted from pages across the
\texttt{calstatela.edu} domain. Version 3.0 expands functionality with Bayesian classification,
improved NLP accuracy, enhanced UI design, and structured SQL database integration.

\subsection{Intended Audience and Reading Suggestions}
This document is intended for developers, testers, marketing staff, project managers, and
documentation authors.  
Readers seeking to understand how the chatbot functions should read the entire document.

Testers may refer to the requirement list and system flow.  
Investors may refer to high-level descriptions and system scope.  
Customers can reference the chatbot capabilities.

\subsection{Product Scope}
The Pineapple Chatbot uses NLP and statistical models to interpret user queries and return the
most relevant webpages or extracted information.

\subsubsection*{User Capabilities}
\begin{itemize}
\item Ask CSULA-related questions
\item Receive text answers or direct website links
\item Obtain professor emails when recognized
\item View responses in a chat-style interface
\end{itemize}

\subsubsection*{System Capabilities}
\begin{itemize}
\item Entity extraction using spaCy
\item Bayesian category scoring for unclear queries
\item Keyword extraction and stop-word removal
\item Ranked URL selection using hybrid scoring
\item SQL database storage of categorized URLs
\item Response generation and formatting
\end{itemize}

\subsection{Definitions, Acronyms, and Abbreviations}
\begin{itemize}
\item \textbf{NLP} – Natural Language Processing
\item \textbf{spaCy} – Python NLP library used for entity recognition
\item \textbf{Entity} – PERSON, ORG, LOCATION, etc.
\item \textbf{Intent} – User’s goal (e.g., find professor email)
\item \textbf{Bayesian Model} – Probability-based classifier
\item \textbf{URL Scoring} – Algorithm selecting the best info source
\item \textbf{CSULA} – California State University, Los Angeles
\end{itemize}

\subsection{References}
\begin{itemize}
\item spaCy Documentation: \url{https://spacy.io}
\item CSULA Website: \url{https://www.calstatela.edu}
\item Python Standard Library
\item Naive Bayes Classification Research
\end{itemize}

\newpage
\section{Overall Description}

\subsection{Product Perspective}
Pineapple Chatbot is a standalone web application supported by:
\begin{itemize}
\item A chat-based frontend UI
\item A Python backend with NLP processing
\item SQL database containing categorized URLs
\end{itemize}

\subsection{Product Functions}

\subsubsection*{User Functions}
\begin{itemize}
\item Submit questions via a text input field
\item View chatbot replies in styled chat bubbles
\item Click links provided in chatbot messages
\item Receive professor-related details such as emails
\end{itemize}

\subsubsection*{System Functions}
\begin{itemize}
\item Extract entities using spaCy
\item Identify intent
\item Reduce keywords and remove stop-words
\item Classify using Bayesian scoring
\item Retrieve URLs using SQL database queries
\item Rank URLs using hybrid scoring
\item Format and send chatbot responses
\end{itemize}

\subsection{User Classes and Characteristics}
\textbf{General Users:} Students or visitors seeking campus information  
\textbf{Developers:} Maintain backend code, NLP pipeline, database, and UI logic  
\textbf{Testers:} Validate reliability, accuracy, and UI behavior  

\subsection{Operating Environment}
\begin{itemize}
\item Python 3.x
\item spaCy NLP models
\item Flask or Node.js backend
\item SQL database (MySQL or SQLite)
\item Modern browsers (Chrome, Firefox, Edge)
\item Optional CSULA VPN for internal hosting
\end{itemize}

\subsection{Design and Implementation Constraints}
\begin{itemize}
\item English-only input supported
\item NLP accuracy limited by spaCy model
\item UI not optimized for mobile
\item Link dataset must be updated periodically
\item Response speed dependent on database size
\end{itemize}

\subsection{User Documentation}
No user guides yet; chatbot will be made public on the CSULA website.

\subsection{Assumptions and Dependencies}
\begin{itemize}
\item CSULA website structure remains stable
\item spaCy NER continues to detect PERSON, ORG, etc.
\item Database remains reachable
\item Questions are written in general English
\end{itemize}

\newpage
\section{External Interface Requirements}

\subsection{User Interfaces}
UI Components include:
\begin{itemize}
\item Chat window
\item Scrollable message history
\item Text input field with send button
\item Styled chat bubbles
\item Highlighted, clickable links
\end{itemize}

\subsection{Hardware Interfaces}
Standard laptop or desktop with internet access.

\subsection{Software Interfaces}
\begin{itemize}
\item spaCy NLP library
\item SQL database with URLs and categories
\item Backend REST API returning JSON
\item Python NLP service
\end{itemize}

\subsection{Communications Interfaces}
Internet access required.  
VPN may be necessary for internal CSULA hosting.  
Not compatible with Internet Explorer.

\newpage
\section{Requirements Specification}

\subsection{Functional Requirements}

\subsubsection*{1. Input Module}
\begin{enumerate}[label=1.\arabic*]
\item System shall accept user text input.
\item System should allow user feedback on responses.
\item System may support multi-sentence input.
\end{enumerate}

\subsubsection*{2. Output Module}
\begin{enumerate}[label=2.\arabic*]
\item System shall display the user's question.
\item System shall include clickable links in responses.
\item System shall display an error for unknown queries.
\end{enumerate}

\subsubsection*{3. Logic Module}
\begin{enumerate}[label=3.\arabic*]
\item Parse keywords from user input.
\item Recognize intent using spaCy.
\item Respond appropriately to unclear or unintelligible inputs.
\item Send keywords and intent to the storage module.
\item Receive possible URLs from storage module.
\item Select the most relevant URL using scoring.
\item Send entities, intent, and keywords to data extraction module.
\end{enumerate}

\subsubsection*{4. Data Extraction Module}
\begin{enumerate}[label=4.\arabic*]
\item Extract information using entities, keywords, and intent.
\item Send extracted information to the output module.
\item Parse and structure text; label using spaCy.
\end{enumerate}

\subsubsection*{5. Storage Module}
\begin{enumerate}[label=5.\arabic*]
\item Send SQL queries.
\item Receive responses from database.
\item Return information to logic module.
\end{enumerate}

\subsection{External Interface Requirements}
\begin{itemize}
\item Input must support standard English text.
\item URL responses must be HTML-safe.
\item Chat must support sequential messaging.
\end{itemize}

\subsection{Logical Database Requirements}
\textbf{Database Tables:}

\subsubsection*{TABLE urls}
\begin{itemize}
\item id
\item url
\item keywords
\item category
\end{itemize}

\subsubsection*{TABLE categories}
\begin{itemize}
\item category\_id
\item name
\item keyword\_distribution
\end{itemize}

\subsubsection*{Requirements}
\begin{itemize}
\item Supports full-text search
\item Allows batch insertion
\item Stores at least 500+ URL entries
\end{itemize}

\subsection{Design Constraints}
\begin{itemize}
\item spaCy model loads on startup
\item Query processing must finish within 3 seconds
\item Category scoring must reach 80–90\% accuracy
\end{itemize}

\newpage
\section{Other Nonfunctional Requirements}

\subsection{Performance Requirements}
\begin{itemize}
\item System must respond within 1–3 seconds
\item Database queries must complete under 200ms
\item UI must remain responsive during processing
\end{itemize}

\subsection{Safety Requirements}
Feedback system must keep reports anonymous to protect user identity.  
Chatbot must avoid exposing sensitive data or giving unsafe guidance.

\subsection{Security Requirements}
\begin{itemize}
\item Backend must sanitize all inputs
\item Database requires authentication
\item Responses must be delivered via HTTPS (future deployment)
\item Admin interface (future) requires login
\end{itemize}

\subsection{Software Quality Attributes}
\begin{itemize}
\item \textbf{Usability:} Simple chat interface
\item \textbf{Reliability:} NLP results consistent per input
\item \textbf{Maintainability:} Modular architecture
\item \textbf{Scalability:} Database supports growth
\end{itemize}

\subsection{Business Rules}
\begin{itemize}
\item Bot must prioritize official CSULA information
\item Answers must originate from the calstatela.edu domain
\item Professor data must be accurate and up-to-date
\end{itemize}

\section{Other Requirements}
Future possibilities include multilingual support and CSULA API integration.

\begin{center}
\textbf{FINAL VERSION 3.0 COMPLETE}
\end{center}

\end{document}
