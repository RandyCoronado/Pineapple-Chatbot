\documentclass[11pt]{article}

\usepackage[margin=1in]{geometry}
\usepackage{hyperref}
\usepackage{longtable}
\usepackage{array}
\usepackage{enumitem}
\usepackage{setspace}

\setlength{\parskip}{0.6em}
\setlength{\parindent}{0pt}
\setcounter{secnumdepth}{3}

\begin{document}

%----------------- TITLE PAGE -----------------%
\begin{titlepage}
    \centering
    \vspace*{3cm}
    {\LARGE \textbf{Software Design Document}\\[0.4cm]
    \textbf{for}\\[0.4cm]
    \textbf{Pineapple Chatbot}\par}
    \vspace{1.5cm}
    {\large Prepared by\\[0.3cm]
    Kuong Thong\\
    Enrique Castillo\\
    Julio Aguilar\\
    Jose Lopez\\
    Yilin Ruan\par}
    \vfill
    {\large CSULA ITS Department\par}
    \vspace*{1cm}
\end{titlepage}

%----------------- VERSION HISTORY -----------------%
\section*{Version History}

\begin{longtable}{>{\raggedright\arraybackslash}p{3cm}%
                  >{\raggedright\arraybackslash}p{3cm}%
                  >{\raggedright\arraybackslash}p{8cm}}
\textbf{Version} & \textbf{Date} & \textbf{Summary of Changes} \\
\hline
apple       & 12/4/18  & Added Introduction sections (1.1--1.2). \\
banana      & 12/5/18  & Added early DFD diagrams and transferred initial design content. \\
cantaloupe  & 12/6/18  & Added database design and sections 8--9.1. \\
dates       & 12/7/18  & Completed document draft and added validation sections. \\
elderberry  & 1/3/19   & Updated DFDs to match current model. \\
farkleberry & 3/13/19  & Updated feature descriptions and removed outdated content. \\
grapefruit  & 3/22/19  & Cleaned document structure and updated sections 5--7. \\
huckleberry & 9/25/19  & Updated crawling approach. \\
jackfruit   & 11/15/19 & Removed Amazon Lex dependency. \\
kiwi        & 12/2/19  & Updated models to reflect spaCy usage. \\
lemon       & 2/12/20  & Added new model with submodules. \\
mango       & 2/19/20  & Switched to Simple Bayesian ranker. \\
nectarine   & 2/26/20  & Fixed duplicate HTML tag entries in the database. \\
orange      & 3/11/20  & Updated front-end for W3C accessibility compliance. \\
\end{longtable}

\newpage
\tableofcontents
\newpage

%----------------- 1. INTRODUCTION -----------------%
\section{Introduction}

\subsection{Purpose}
Pineapple Chatbot is an informational chatbot for a standalone web application that helps users find information related to the CSULA website. The chatbot processes user queries, responds with useful answers, and can receive user feedback.

\subsection{Intended Audience}
This document is intended for developers, testers, designers, project managers, and documentation staff involved in the development, maintenance, and operation of the Pineapple Chatbot system.

\subsection{System Overview}
The Pineapple Chatbot system crawls university webpages, stores extracted information in a database, and uses natural-language processing (NLP) to interpret user queries. It returns answers to users as text, links, or summaries through a web-based chat interface.

%----------------- 2. DESIGN CONSIDERATIONS -----------------%
\section{Design Considerations}

\subsection{Assumptions and Dependencies}
\begin{itemize}[nosep]
    \item Backend is built using Python.
    \item Core modules and libraries include:
    \begin{itemize}[nosep]
        \item MySQL Connector
        \item BeautifulSoup
        \item \texttt{nltk}
        \item ChatterBot
        \item spaCy
    \end{itemize}
    \item Other dependencies:
    \begin{itemize}[nosep]
        \item NodeJS
        \item React (front-end)
    \end{itemize}
\end{itemize}

\subsection{Constraints}
\begin{itemize}[nosep]
    \item Early versions of the system lack spell-check, web abstraction, and high-accuracy NLP capabilities.
    \item Overall system quality depends heavily on the data gathered by the crawler; vague or ambiguous user queries may produce incorrect or low-quality results.
\end{itemize}

\subsection{Goals}
\begin{itemize}[nosep]
    \item Provide helpful, conversational responses to user queries.
    \item Keep the design simple, following the ``Keep It Simple, Stupid'' (KISS) principle.
    \item Provide useful detail without overwhelming users with excessive information.
    \item Support user satisfaction through clear, understandable output.
\end{itemize}

\subsection{Development Methods}
Pair programming and iterative testing were used to refine modules and improve response accuracy. The team followed an incremental approach, integrating and evaluating components frequently in order to improve robustness and usability.

%----------------- 3. ARCHITECTURAL STRATEGY -----------------%
\section{Architectural Strategy}

Python is used for crawling, NLP, and chatbot logic due to its strong open-source ecosystem and availability of mature libraries for text processing and web scraping. MySQL is used to store processed data because the ChatterBot framework integrates well with SQL-based storage and provides efficient data retrieval.

The chatbot runs inside a web application built with React. This enables a responsive text-based user interface where users can enter queries and view responses within a browser.

%----------------- 4. SYSTEM ARCHITECTURE -----------------%
\section{System Architecture}

The system consists of five major modules:
\begin{enumerate}[nosep]
    \item \textbf{Input Module} -- Collects user inputs.
    \item \textbf{Preprocessor Module} -- Cleans and normalizes text.
    \item \textbf{Logic Module} -- Identifies user intent and keywords, and selects an appropriate response.
    \item \textbf{Storage Module} -- Stores and retrieves crawled data and responses.
    \item \textbf{Output Module} -- Formats and returns the final answer to the user.
\end{enumerate}

\subsection*{Data Flow Summary}
The high-level data flow of the system can be summarized as follows:
\begin{enumerate}[nosep]
    \item User submits a query via the user interface.
    \item The query is preprocessed (cleaning, normalization).
    \item Intent recognition and keyword identification are performed.
    \item Relevant data is retrieved from the storage module.
    \item The system constructs and returns a response to the user.
\end{enumerate}

%----------------- 5. POLICIES AND TACTICS -----------------%
\section{Policies and Tactics}

\subsection{Tools and Technologies}
The primary tools and technologies used in the system include:
\begin{itemize}[nosep]
    \item ChatterBot
    \item React
    \item Python
    \item MySQL
\end{itemize}

\subsection{Requirements Traceability}
Crawled data is backed up regularly to support requirements for reliability and recovery. This ensures that information used by the chatbot can be restored in case of failures or data corruption.

\subsection{Testing Approach}
User testing and quality assurance (QA) are employed to verify the accuracy and relevance of responses. Test cases cover both the crawling and NLP components, as well as the integrated end-to-end chatbot experience.

\subsection{Engineering Trade-offs}
Data is stored from many different campus websites in order to improve coverage and answer a broad range of questions. This is balanced against response speed and storage size. Design choices aim to maintain acceptable performance while maximizing informational coverage.

\subsection{Coding Guidelines}
Standard Python and JavaScript coding conventions are followed. Version control is used throughout the project to manage changes, facilitate collaboration, and track the evolution of the codebase.

\subsection{Source Code Organization}
\begin{itemize}[nosep]
    \item \textbf{React App:} User interface components.
    \item \textbf{Python Backend:} Chatbot logic, server-side processing, and integration with the database.
    \item \textbf{Database:} MySQL tables and associated schemas.
\end{itemize}

\subsection{System Build Instructions (Abbreviated)}
\begin{enumerate}[nosep]
    \item Start the virtual machine (VM) or server hosting the application.
    \item Run the React frontend using:
    \begin{quote}
        \texttt{npm start}
    \end{quote}
    \item Run the Python backend using:
    \begin{quote}
        \texttt{python manage.py runserver}
    \end{quote}
    \item Access the web interface at the configured CSULA server URL (when available).
\end{enumerate}

%----------------- 6. DETAILED SYSTEM DESIGN -----------------%
\section{Detailed System Design}

\subsection{Input Module}
\textbf{Responsibilities:}
\begin{itemize}[nosep]
    \item Convert user input (text, speech, or terminal input) into \texttt{Statement} objects used by the chatbot.
\end{itemize}

\textbf{Constraints:}
\begin{itemize}[nosep]
    \item Must handle plain text and minimal structured formats.
\end{itemize}

\textbf{Interactions:}
\begin{itemize}[nosep]
    \item Interacts directly with user input and passes processed text or \texttt{Statement} objects to the Logic Module.
\end{itemize}

\subsection{Preprocessor Module}
\textbf{Responsibilities:}
\begin{itemize}[nosep]
    \item Clean whitespace.
    \item Remove HTML artifacts.
    \item Convert Unicode characters to ASCII where appropriate.
\end{itemize}

\textbf{Limitations:}
\begin{itemize}[nosep]
    \item Must preserve important keywords while removing noise and irrelevant symbols.
\end{itemize}

\subsection{Storage Module}
\textbf{Responsibilities:}
\begin{itemize}[nosep]
    \item Store responses and crawled data.
    \item Support efficient queries for keyword and URL lookup.
\end{itemize}

\textbf{Technology:}
\begin{itemize}[nosep]
    \item MySQL database.
\end{itemize}

\textbf{Built-in Options:}
\begin{itemize}[nosep]
    \item ChatterBot \texttt{SQLStorageAdapter}
    \item ChatterBot \texttt{MongoDatabaseAdapter} (alternative)
\end{itemize}

\textbf{Limitations:}
\begin{itemize}[nosep]
    \item Requires valid user credentials and appropriate access permissions.
    \item Large datasets may slow queries and require optimization or indexing.
\end{itemize}

\subsection{Logic Module}
\textbf{Responsibilities:}
\begin{itemize}[nosep]
    \item Recognize user intent.
    \item Identify keywords in queries.
    \item Select the best-matched response based on similarity or ranking.
    \item Support time-based and mathematical query evaluation where applicable.
\end{itemize}

\textbf{Limitations:}
\begin{itemize}[nosep]
    \item Accuracy depends on the available training data and quality of crawled content.
    \item Complex or ambiguous phrasing may reduce match accuracy and lead to suboptimal responses.
\end{itemize}

\subsection{Output Module}
\textbf{Responsibilities:}
\begin{itemize}[nosep]
    \item Format and return the final response to the user via the React frontend.
\end{itemize}

\textbf{Possible Output Mechanisms:}
\begin{itemize}[nosep]
    \item Terminal-based output (for development or debugging).
    \item API-based output for integration with other systems (if extended).
\end{itemize}

%----------------- 7. LOWER-LEVEL COMPONENT DESIGN -----------------%
\section{Lower-Level Component Design}

\subsection{Input Module Components}
\begin{itemize}[nosep]
    \item Creates \texttt{Statement} objects from input.
    \item Serves as the base class for all input types.
    \item Only text, dictionary, or \texttt{Statement} formats are supported.
\end{itemize}

\subsection{Output Module Components}
\begin{itemize}[nosep]
    \item Returns \texttt{Statement} objects as responses.
    \item Reads data from the Storage Module before formatting it for display.
    \item Has minimal performance constraints since most computation occurs in the Logic Module.
\end{itemize}

\subsection{Storage Module Components}
\begin{itemize}[nosep]
    \item Performs all CRUD (Create, Read, Update, Delete) operations for stored responses and crawled data.
    \item Requires valid database connection credentials.
    \item Uses the SQL-based module as the primary implementation.
\end{itemize}

\subsection{Logic Module Components}
\begin{itemize}[nosep]
    \item Compares user statements with known statements stored in the database.
    \item Uses similarity thresholds, maximum similarity checks, and exclusion rules to rank candidate responses.
    \item Selects the closest match and retrieves an appropriate response from the database.
\end{itemize}

%----------------- 8. DATABASE DESIGN -----------------%
\section{Database Design}

A MySQL database stores the crawled website data and associated metadata.

\subsection*{Key Table: \texttt{termUrlKeywords}}
This table contains approximately 300{,}000 records and includes the following columns:
\begin{itemize}[nosep]
    \item \textbf{term} -- The search term or key phrase.
    \item \textbf{URL} -- The target URL associated with the term.
    \item \textbf{keywords} -- Additional keywords related to the term and URL.
\end{itemize}

Additional tables store faculty information and labels for the Bayesian classifier used in query interpretation and ranking.

The database schema is designed to enable fast keyword matching, URL lookup, and long-term storage of crawled data.

%----------------- 9. USER INTERFACE -----------------%
\section{User Interface}

\subsection{Overview}
The user interface is a simple, chat-style interface built with React. It provides a conversational experience where users can enter queries and view responses within a web application.

\subsection{Screens and Layouts}
Key UI elements include:
\begin{itemize}[nosep]
    \item An input box where users type their queries.
    \item A response window where the chatbot displays answers, links, and summaries.
\end{itemize}

\subsection{UI Flow Model}
\begin{enumerate}[nosep]
    \item User opens the web application.
    \item User enters a question or statement in the input box.
    \item The system processes the input and calls backend APIs.
    \item The chatbot generates a response, which is displayed in the response window.
    \item Optional: user provides feedback or continues the conversation.
\end{enumerate}

Future enhancements may include richer UI components, additional accessibility improvements, and support for more interactive elements.

%----------------- 10. VALIDATION AND VERIFICATION -----------------%
\section{Requirements Validation and Verification}

To ensure correctness, reliability, and usability, the following testing activities are performed:
\begin{itemize}[nosep]
    \item \textbf{Unit tests} for crawler components, NLP functions, and storage operations.
    \item \textbf{Integration tests} for the end-to-end query flow from user input to response.
    \item \textbf{User testing} to verify clarity, relevance, and accuracy of responses in realistic scenarios.
\end{itemize}

%----------------- 11. GLOSSARY -----------------%
\section{Glossary}

\textbf{Statement} \\
Object representing a processed user input used by the chatbot for comparison and response selection.

\textbf{Intent} \\
The interpreted purpose or goal of a user query (e.g., asking for information, requesting a URL, etc.).

\textbf{Similarity Score} \\
A metric used to choose the closest matching response to a given user query based on stored statements and training data.

%----------------- 12. REFERENCES -----------------%
\section{References}

\begin{itemize}[nosep]
    \item Original Pineapple Chatbot design documents.
    \item Tutorials and documentation for Python NLP libraries, including ChatterBot, spaCy, \texttt{nltk}, and BeautifulSoup.
    \item MySQL documentation for database configuration, optimization, and query techniques.
\end{itemize}

\end{document}
