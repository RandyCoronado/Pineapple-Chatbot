\documentclass[11pt]{article}

\usepackage[margin=1in]{geometry}
\usepackage{hyperref}
\usepackage{longtable}
\usepackage{array}
\usepackage{enumitem}
\usepackage{setspace}
\usepackage{graphicx}

\setlength{\parskip}{0.6em}
\setlength{\parindent}{0pt}
\setcounter{secnumdepth}{3}

\begin{document}

%----------------- TITLE PAGE -----------------%
\begin{titlepage}
    \centering
    \vspace*{3cm}

    {\Large \textbf{Software Design Document}\par}
    \vspace{0.5cm}
    {\LARGE \textbf{Pineapple Chatbot}\par}

    \vspace{1.5cm}
    {\large Prepared by\par}
    \vspace{0.5cm}
    {\large Kuong Thong, Enrique Castillo, Julio Aguilar, Jose Lopez, Yilin Ruan\par}

    \vfill
    {\large CSULA ITS Department\par}
    \vspace*{1cm}
\end{titlepage}

%----------------- VERSION HISTORY -----------------%
\section*{Version History}

\begin{longtable}{>{\raggedright\arraybackslash}p{3cm}%
                  >{\raggedright\arraybackslash}p{3cm}%
                  >{\raggedright\arraybackslash}p{8cm}}
\textbf{Version} & \textbf{Date} & \textbf{Summary of Changes} \\
\hline
apple      & 12/4/18 & Added Introduction sections (1.1--1.2). \\
banana     & 12/5/18 & Added early DFD diagrams and transferred initial design content. \\
cantaloupe & 12/6/18 & Added database design and sections 8--9.1. \\
dates      & 12/7/18 & Completed document draft and added validation sections. \\
\end{longtable}

\newpage
\tableofcontents
\newpage

%----------------- 1. INTRODUCTION -----------------%
\section{Introduction}

\subsection{Purpose}
Pineapple Chatbot is an informational chatbot for a standalone web application that helps
users find information related to the CSULA website. The chatbot processes user queries,
responds with useful answers, and can receive user feedback.

\subsection{Document Conventions}
% Add any document style or notation conventions here.
This document follows a standard Software Design Document (SDD) structure with numbered
sections and subsections.

\subsection{Intended Audience}
This document is for developers, testers, designers, project managers, and documentation staff.

\subsection{System Overview}
Pineapple Chatbot receives natural-language queries from users, analyzes the text to determine
intent and relevant keywords, retrieves matching information from a backend data store, and
returns an appropriate response.

%----------------- 4. SYSTEM ARCHITECTURE -----------------%
\section{System Architecture}

\subsection{Level 0 DFD}
\begin{figure}[h!]
    \centering
    % Replace the filename below with the actual Level 0 DFD image file
    \includegraphics[width=0.8\textwidth]{DFD_Level0_PineappleChatbot}
    \caption{DFD -- Level 0 for Pineapple Chatbot}
\end{figure}

At a high level, the system receives user input, processes it through internal modules, and
returns a response.

\subsection{Level 1 DFD}
\begin{figure}[h!]
    \centering
    % Replace the filename below with the actual Level 1 DFD image file
    \includegraphics[width=\textwidth]{DFD_Level1_PineappleChatbot}
    \caption{DFD -- Level 1 for Pineapple Chatbot}
\end{figure}

The system consists of five major modules:
\begin{enumerate}[nosep]
    \item \textbf{Input Module} -- Collects user inputs.
    \item \textbf{Preprocessor Module} -- Cleans and normalizes text.
    \item \textbf{Logic Module} -- Identifies user intent, keywords, and selects a response.
    \item \textbf{Storage Module} -- Stores and retrieves crawled data and responses.
    \item \textbf{Output Module} -- Formats and returns the final answer to the user.
\end{enumerate}

\subsection*{Data Flow Summary}
User query $\rightarrow$ Preprocessing $\rightarrow$ Intent recognition $\rightarrow$ Keyword lookup
$\rightarrow$ Data retrieval $\rightarrow$ Response $\rightarrow$ User.

%----------------- 8. DATABASE DESIGN -----------------%
\section{Database Design}

A MySQL database stores the crawled website data and related information required by the
chatbot.

\subsection*{Key Table: \texttt{termUrlKeywords} (approximately 300{,}000 records)}
\begin{itemize}[nosep]
    \item \textbf{Columns:}
    \begin{itemize}[nosep]
        \item \texttt{term}
        \item \texttt{URL}
        \item \texttt{keywords}
    \end{itemize}
\end{itemize}

Additional tables store faculty information and labels for the Bayesian classifier.

The database enables:
\begin{itemize}[nosep]
    \item Fast keyword matching
    \item Efficient URL lookup
    \item Long-term storage of crawled data
\end{itemize}

%----------------- 9. USER INTERFACE -----------------%
\section{User Interface}

The Pineapple Chatbot user interface is designed as a simple, web-based chat interface.

\subsection*{Overview}
\begin{itemize}[nosep]
    \item Simple chat-style interface built with React.
    \item Input box for user queries.
    \item Response window for answers and links.
\end{itemize}

\subsection*{Future Enhancements}
\begin{itemize}[nosep]
    \item Richer UI components (e.g., quick-reply buttons, suggested queries).
    \item Improved accessibility and compliance with web accessibility standards.
\end{itemize}

%----------------- 11. GLOSSARY -----------------%
\section{Glossary}

\begin{itemize}[nosep]
    \item \textbf{Statement}: Object representing a processed user input.
    \item \textbf{Intent}: The interpreted purpose of a user query.
    \item \textbf{Similarity Score}: Metric used to choose the closest matching response.
\end{itemize}

%----------------- 12. REFERENCES -----------------%
\section{References}

Original Pineapple Chatbot design documents, tutorials for Python NLP libraries, and MySQL
documentation.

\end{document}
